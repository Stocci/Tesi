% !TEX encoding = UTF-8
% !TEX TS-program = pdflatex
% !TEX root = ../tesi.tex

%**************************************************************
\chapter{Descrizione dello stage}
\label{cap:descrizione-stage}
%**************************************************************

\intro{In questo capitolo descriverò in maniera approfondita lo scopo dello stage, focalizzandomi sugli obiettivi da raggiungere e le metodologie per farlo.\\Seguirà poi una descrizione dettagliata dei casi d'uso di interesse.}\\

%**************************************************************
\section{Introduzione al progetto}
%TODO serve riprenderlo ancora?
%**************************************************************

\section{Dominio applicativo}
\subsection{Tipi di utenti}
JGalileo CRM è stato pensato per essere utilizzato da svariate tipologie di utenti:
i casi d'uso seguiranno solamente i due utenti più importanti:
\begin{itemize}
	\item \textbf{amministratore}: che ha il potere di creare nuovi utenti e configurare l'interfaccia finale aggiungendo e togliendo \gls{portletg}. Inoltre può accedere al pannello di controllo e monitorare la lista di utenti, modificarne i permessi e visualizzare svariate statistiche di utilizzo del prodotto;
	\item{utente semplice}, cui è permesso solamente utilizzare il software, senza aggiungere o togliere \gls{portletg} e naturalmente senza avere la possibilità di creare altri utenti.\\
\end{itemize}
	Quest'ultima è la tipologia di utenti più comune, infatti la clientela di JGalileo CRM è composta solamente da utenti semplici, mentre solamente i membri del team di sviluppo possiedono i permessi di amministratore.
\subsection{Funzionalità}
Le funzionalità prese in esame si sviluppano soprattutto sull'interazione tra l'utente semplice e la form la cui creazione era l'obiettivo del mio stage.
Quindi i casi d'uso riguarderanno principalmente:
\begin{itemize}
	\item \textbf{Inserimento di un nuova categoria di clienti};
	\item \textbf{Inserimento di attività ed opportunità};
	\item \textbf{visualizzazione dettagliata dei clienti già inseriti};
	\item \textbf{visualizzazione dettagliata delle attività ed opportunità già inserite};
	\item \textbf{modifica e salvataggio dei dati};
	\item \textbf{invio di email, fax e newsletter}.
\end{itemize}
In particolare, le categorie di clienti inseribili tramite form sono:
\begin{itemize}
	\item \textbf{Leads}: Sono persone ed aziende che possono avere interesse ai prodotti e servizi dell'utente. Anche un primo contatto, come ad esempio un biglietto da visita, è considerato un lead;
	\item \textbf{Accounts}: Contengono i dati di un'azienda con cui è già presente un qualche tipo di relazione commerciale, anche solo una roposta d'acquisto;
	\item \textbf{Contatti}: I contatti consistono in persone fisiche con cui è già presente un qualche tipo di relazione commerciale. Spesso account e contatto sono in relazione tra di loro, ma è anche possibile che un contatto non abbia nessun account associato (ad esempio un privato).
\end{itemize}
Per quanto riguarda la parte di amministrazione, verrà illustrata solamente la procedura di inserimento di un nuovo utente.

%**************************************************************
\section{Casi d'uso}
I casi d'uso (use case) sono una tecnica dell'ingegneria del software per ottenere un'analisi precisa e senza ambiguità dei requisiti di un sistema, con l'obiettivo di perseguire la creazione di software di qualità.\\
I casi d'uso sono solitamente rappresentati in forma testuale, attraverso la descrizione dettagliata del caso d'uso stesso in termini di attori coinvolti, pre e post condizioni, e da una rappresentazione grafica, definita Use Case Diagram, con l'ausislio di un altro linguaggio, \gls{umlg}.\\
Tutti i casi d'uso saranno qui presentati attraverso descrizione testuale, mentre la rappresentazione grafica verrà utilizzata solamente per gli scenari più generali.\\
I casi d'uso qui riportati non si riferiscono all'intero software a causa della sua grande complessità ed ampiezza d'utilizzo, ma solamente agli scenari principali che, direttamente od indirettamente, sono influenzati dal codice da me scritto durante l'esperienza di stage.\\
\subsection{Classificazione dei casi d'uso}
Ogni caso d’uso è classificato secondo la seguente convenzione:
\begin{center}
	UC[codice]
\end{center}
UC[codice]
Dove [codice] è un codice numerico che identifica univocamente il caso d’uso.\\
Esso è sequenziale e gerarchico, dunque se ad esempio il caso d’uso "X" necessita di un ulteriore livello di dettaglio si procede individuando i sotto-casi d’uso "X.Y", "X.Y.Z" etc.. dove:\\
\begin{itemize}
	\item \textbf{X}: il codice identificativo del caso d'uso padre, solitamente generico e difficilemte descrivibile nel dettaglio; 
	\item \textbf{Y}: codice figlio, identifica un caso d'uso più particolare rispetto al padre (che lo contiene).
	\item \textbf{Z}: codice nipote, individua lo scenario più particolare possibile.
	
\end{itemize}
X parte con indice 1 e valore crescente, mentre Y e Z parto dall'indice 0 con valore crescente.\\
Indicherò con il suffisso “\_G” tutti quei casi d’uso di alto livello, a segnalare che si tratta di una visione generale del contesto.
\subsection{Struttura dei casi d'uso}
Ogni caso d'uso verrà descritto in maniera testuale, utilizzando la seguente struttura:
\begin{itemize}
	\item \textbf{Attori}: Gli attori comprendono le entità umane che interagiscono con il sistema. \\ 
	Nella fattispecie, vengono distinti tre tipologie di attori:
	\begin{itemize}
		\item \textbf{Amministratore}: Questa tipologia di attore è quella riservata solamente agli sviluppatori del prodotto;
		\item \textbf{Utente non autenticato}: indica un attore non ancora autenticato nel sistema;
		\item \textbf{Utente autenticato}, indicato anche semplicemente come \textbf{Utente}, indica un utente non amministratore che viene riconosciuto dal sistema e può quindi usufruirne.
	\end{itemize}
	\item \textbf{Pre-condizione}: Indica le condizioni che devono necessariamente essere soddisfatte affinchè sia possibile l'interazione dell'attore con lo specifico caso d'uso;
	\item \textbf{Post-condizione}: Indica le condizioni in cui l'attore si verrà a trovare dopo la fine dell'interazione con lo specifico caso d'uso;
	\item \textbf{Descrizione}: Riporta una breve descrizione testuale del caso d'uso.
\end{itemize}
\subsection{Funzionalità amministratore}
\begin{figure}[h]
	\centering
	\includegraphics[height = 8 cm]{/usecase/funzionalità-amm}
	\caption{Caso d'uso generale-funzionalità amministratore}
\end{figure}

\subsubsection{UC 1}

\begin{itemize}
	\item \textbf{Descrizione}: l'amministratore può aggiungere un nuovo utente inserendo questi campi in una form:
	\begin{itemize}
		\item %TODO campi da inserire per aggiungere utente
	\end{itemize}
	\item \textbf{Attore}: Utente amministratore;
	\item \textbf{Pre-condizione}: L'amministratore ha già effettuato l'accesso tramite inserimento di username e password, ed ha raggiunto il pannello di amministrazione;
	\item \textbf{Post-condizione}: è stato inserito un nuovo utente.
\end{itemize}

\subsection{Funzionalità utente}
\begin{figure}[h]
	\centering
	\includegraphics[height = 12 cm]{/usecase/funzionalità-utente}
	\caption{Caso d'uso generale-funzionalità utente}
\end{figure}
\subsubsection{UC 2 - Autenticazione}

\begin{itemize}
	\item \textbf{Descrizione}: per l'autenticazione è necessario inserire nel sistema:
	\begin{itemize}
		\item \textbf{nome utente};
		\item \textbf{password}.
	\end{itemize}
	\item \textbf{Attore}: Utente non autenticato;
	\item \textbf{Pre-condizione}: Un utente amministratore deve aver aggiunto l'utente che sta per effettuare l'accesso affinchè l'autenticazione vada a buon fine;
	\item \textbf{Post-condizione}:l'utente viene autenticato e può utilizzarne le funzionalità del prodotto in base ai suoi permessi.
\end{itemize}

\subsubsection{UC 3 - Aggiunta lead}

\begin{itemize}
	\item \textbf{Descrizione}: un utente può aggiungere un lead al suo insieme di potenziali clienti. Per farlo deve inserire alcune informazioni obbligatorie, come:
	\begin{itemize}
		\item \textbf{cognome};
		\item \textbf{nome della società}.
	\end{itemize}
	e molte altre facoltative, ma che sono molto importanti al fine di avere una rapida e completa panoramica del potenziale cliente. Tra queste ci sono:
	\begin{itemize}
		\item \textbf{email};
		\item \textbf{indirizzo dell'ufficio};
		\item \textbf{descrizione del lead};
		\item \textbf{numero di cellulare};
		\item \textbf{provenienza del lead}.
	\end{itemize}
	\item \textbf{Attore}: Utente autenticato;
	\item \textbf{Pre-condizione}: Un utente deve aver eseguito l'accesso ed essersi portato alla pagina di inserimento lead;
	\item \textbf{Post-condizione}:il nuovo lead contenente le informazioni inserite viene salvato nel sistema ed è disponibile all'utente.
\end{itemize}

\subsubsection{UC 4 - Aggiunta account}

\begin{itemize}
	\item \textbf{Descrizione}: un utente può aggiungere un account alla lista di clienti. Un account può essere di vari tipi, come:
	\begin{itemize}
		\item \textbf{Prospect} (UC 4.1): un prospect rappresenta una potenziale azienda cliente, con la quale non si è ancora aperto alcun tipo di rapporto commerciale;
		\item \textbf{Partner} (UC 4.2):un partner consiste in un'azienda con cui si è stabilito un duraturo rapporto di scambi, eventualmente anche non solo commerciali;
		\item \textbf{Organizzazione} (UC 4.3):%TODO;
		\item \textbf{Concorrente} (UC 4.4): questo tipo di account si riferisce ad un'azienda che opera nello stesso settore del cliente.
	\end{itemize}
   Questi tipi di account hanno form che utilizzano campi dati anche molto diversi tra loro, %TODO completare
	e molte altre facoltative, tra le quali:
	\begin{itemize}
		\item \textbf{nome utente};
		\item \textbf{password}.
	\end{itemize}
	\item \textbf{Attore}: Utente autenticato;
	\item \textbf{Pre-condizione}: Un utente deve aver eseguito l'accesso ed essersi portato alla pagina di inserimento account;
	\item \textbf{Post-condizione}:il nuovo account contenente le informazioni inserite viene salvato nel sistema ed è disponibile all'utente.
\end{itemize}

\subsubsection{UC 5 - Aggiunta contatto}

\begin{itemize}
	\item \textbf{Descrizione}: un utente può aggiungere un contatto al sistema. Per farlo deve inserire alcune informazioni obbligatorie, come:
	\begin{itemize}
		\item \textbf{cognome};
		\item \textbf{account da associare al contatto}.
	\end{itemize}
	e molte altre facoltative, tra le quali:
	\begin{itemize}
		\item \textbf{email};
		\item \textbf{indirizzo dell'ufficio};
		\item \textbf{qualifica del contatto};
		\item \textbf{stabilimento di lavoro del contatto}.
	\end{itemize}
	\item \textbf{Attore}: Utente autenticato;
	\item \textbf{Pre-condizione}: Un utente deve aver eseguito l'accesso ed essersi portato alla pagina di inserimento contatto;
	\item \textbf{Post-condizione}:il nuovo contatto contenente le informazioni inserite viene salvato nel sistema ed è disponibile all'utente.
\end{itemize}

\subsubsection{UC 6 - Aggiunta attività}

\begin{itemize}
	\item \textbf{Descrizione}: un utente può aggiungere un'attività alla lista di clienti. Per farlo deve inserire alcune informazioni obbligatorie, come:
	\begin{itemize}
		\item \textbf{nome utente};
		\item \textbf{password}.
	\end{itemize}
	e molte altre facoltative, tra le quali:
	\begin{itemize}
		\item \textbf{nome utente};
		\item \textbf{password}.
	\end{itemize}
	\item \textbf{Attore}: Utente autenticato;
	\item \textbf{Pre-condizione}: Un utente deve aver eseguito l'accesso ed essersi portato alla pagina di inserimento account;
	\item \textbf{Post-condizione}:il nuovo account contenente le informazioni inserite viene salvato nel sistema ed è disponibile all'utente.
\end{itemize}

\subsubsection{UC 7 - Aggiunta opportunità}


\begin{itemize}
	\item \textbf{Descrizione}: un utente può aggiungere un'opportunità alla lista di clienti. Per farlo deve inserire alcune informazioni obbligatorie, come:
	\begin{itemize}
		\item \textbf{nome utente};
		\item \textbf{password}.
	\end{itemize}
	e molte altre facoltative, tra le quali:
	s\begin{itemize}
		\item \textbf{nome utente};
		\item \textbf{password}.
	\end{itemize}
	\item \textbf{Attore}: Utente autenticato;
	\item \textbf{Pre-condizione}: Un utente deve aver eseguito l'accesso ed essersi portato alla pagina di inserimento account;
	\item \textbf{Post-condizione}:il nuovo account contenente le informazioni inserite viene salvato nel sistema ed è disponibile all'utente.
\end{itemize}