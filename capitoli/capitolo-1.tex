% !TEX encoding = UTF-8
% !TEX TS-program = pdflatex
% !TEX root = ../tesi.tex

%**************************************************************
\chapter{Introduzione}
\label{cap:introduzione}
%**************************************************************

Introduzione al contesto applicativo.\\

\noindent Esempio di utilizzo di un termine nel glossario \\
\gls{api}. \\

\noindent Esempio di citazione in linea \\
\cite{site:agile-manifesto}. \\


%**************************************************************
\section{L'azienda}



\begin{figure}[h]
\centering
\includegraphics[height = 5 cm]{sanmarco-logo}
\caption{Logo di Sanmarco Informatica}
\end{figure}
Sanmarco Informatica Spa è un'azienda italiana leader nella progettazione e realizzazione di soluzioni a supporto della riorganizzazione di tutti i processi aziendali e professionali. \\ Ha sede a  Grisignano di Zocco (VI), è attiva da 35 anni, può contare su più di 400 dipendenti e collaboratori, per seguire quotidianamente più di 1000 aziende clienti. \\ 
Il punto di forza dell'azienda sta nella formazione del personale e nella ricerca, campi nei quali viene investito in media il 20\% del fatturato annuo. \\,
Il prodotto di punta di  Sanmarco Informatica è \emph{JGalileo}, un software gestionale \emph{ERP}\glsfirstoccur utilizzato, oltre che da aziende europee, anche in USA, Russia e Cina.\\

Nel centro di Sviluppo dell'azienda viene adottato il metodo Agile: esso nasce nel 2001 e definisce i dodici principi fondamentali per sviluppare con tempistiche e procedure ottimizzate un software che sia per il cliente una soluzione di successo. La metodologia “Agile” privilegia \emph{gli individui e le interazioni più che i processi e gli strumenti}, \emph{il software funzionante più che la documentazione esaustiva}, \emph{la collaborazione col cliente più che la negoziazione dei contratti}, \emph{rispondere al cambiamento più che seguire un piano}. \\ L’obiettivo è quello di garantire all’azienda tutta la flessibilità e la versatilità necessaria alla realizzazione di progetti efficaci. Questo metodo di lavoro si concentra su una collaborazione più face-to-face fra colleghi con cadenza regolare e ravvicinata, permettendo di mantenere e perfezionare man mano il focus sui progetti, valorizzando i nuovi punti di vista e le nuove risposte alle problematiche da affrontare. \\ In particolare, il metodo di gestione progetti “SCRUM” prevede di dividere il progetto in blocchi (Sprint), all’interno dei quali vengono sviluppate delle funzioni/moduli/applicazioni complete (denominate storie) pronte per la potenziale installazione al cliente. Il termine Scrum è mutuato dal termine del Rugby che indica il pacchetto di mischia ed è una metafora del team di sviluppo che deve lavorare insieme in modo che tutti gli attori del progetto spingano nella stessa direzione, agendo come un’unica entità coordinata.


%**************************************************************
\section{L'offerta di Stage}
Da anni l'azienda collabora con l'università di Padova alla ricerca di neolaureati da formare ed inserire nel proprio organico e, grazie ad iniziative come StageIt le possibilità di incontro tra studenti ed aziende sono molto più concrete e produttive.\\
\begin{figure}[h]
\centering
\includegraphics[height = 4 cm]{stageit2018}
\caption{Logo di StageIT}
\end{figure}\\ 
Gli studenti hanno infatti l' occasione di conoscere svariate realtà, anche piccole, che operano nel settore nei più disparati rami applicativi mentre per le aziende, oltre ad essere un'importante vetrina ed un momento di confronto con potenziali partner e concorrenti, costituisce anche la possibilità di conoscere tanti giovani studenti in un breve arco temporale.\\
Tra le varie proposte di stage che quest'azienda offriva, mi è stato proposto di collaborare a quello riguardante il loro software CRM.\\
Un software CRM, acronimo di Costumer Relationship Management, è un prodotto che permette all'utilizzatore di mantenere una rete di relazioni con clienti e potenziali clienti, non soltanto al fine di mantenere una relazione commerciale ma di fiducia reciproca che si possa protrarre nel tempo, anche attraverso l'impiego di strumenti di fidelizzazione come newsletters, campagne ed eventi. \\
In questo contesto si inserisce  lo stage cui ho preso parte e che aveva come obiettivo la ridefinizione della form principale di tale software, dapprima realizzato in \emph{Java} con traduzione in \emph{JavaScript} a cura della libreria \emph{GWT} ed ora richiesto in \emph{JavaScript} nativo, utilizzando una libreria da definire in fase di svolgimento dello stage dopo un'adeguata analisi delle alternative open-source disponibili sul mercato\footcite{queste ed altre tecnologie verranno discusse nel dettaglio nei capitoli seguenti}. 

%**************************************************************
