% !TEX encoding = UTF-8
% !TEX TS-program = pdflatex
% !TEX root = ../tesi.tex

%**************************************************************
\chapter{Introduzione}
\label{cap:introduzione}
%**************************************************************

Introduzione al contesto applicativo.\\

\noindent Esempio di utilizzo di un termine nel glossario \\
\gls{api}. \\

\noindent Esempio di citazione in linea \\
\cite{site:agile-manifesto}. \\


%**************************************************************
\section{L'azienda}



\begin{figure}[h]
\centering
\includegraphics[height = 5 cm]{sanmarco-logo}
\caption{Logo di Sanmarco Informatica}
\end{figure}
Sanmarco Informatica Spa è un'azienda italiana leader nella progettazione e realizzazione di soluzioni a supporto della riorganizzazione di tutti i processi aziendali e professionali. \\ Ha sede a  Grisignano di Zocco (VI), è attiva da 35 anni, può contare su più di 400 dipendenti e collaboratori, per seguire quotidianamente più di 1000 aziende clienti. \\ 
Il punto di forza dell'azienda sta nella formazione del personale e nella ricerca, campi nei quali viene investito in media il 20\% del fatturato annuo. \\,
Il prodotto di punta di  Sanmarco Informatica è \emph{JGalileo}, un software gestionale \emph{ERP}\glsfirstoccur utilizzato, oltre che da aziende europee, anche in USA, Russia e Cina.\\

Nel Centro di Sviluppo dell'azienda viene adottato il metodo Agile: esso nasce nel 2001 e definisce i dodici principi fondamentali per sviluppare con tempistiche e procedure ottimizzate un software che sia per il cliente una soluzione di successo. La metodologia “Agile” privilegia \emph{gli individui e le interazioni più che i processi e gli strumenti}, \emph{il software funzionante più che la documentazione esaustiva}, \emph{la collaborazione col cliente più che la negoziazione dei contratti}, \emph{rispondere al cambiamento più che seguire un piano}. \\ L’obiettivo è quello di garantire all’azienda tutta la flessibilità e la versatilità necessaria alla realizzazione di progetti efficaci. Questo metodo di lavoro si concentra su una collaborazione più face-to-face fra colleghi con cadenza regolare e ravvicinata, permettendo di mantenere e perfezionare man mano il focus sui progetti, valorizzando i nuovi punti di vista e le nuove risposte alle problematiche da affrontare. \\ In particolare, il metodo di gestione progetti “SCRUM” prevede di dividere il progetto in blocchi (Sprint), all’interno dei quali vengono sviluppate delle funzioni/moduli/applicazioni complete (denominate storie) pronte per la potenziale installazione al cliente. Il termine Scrum è mutuato dal termine del Rugby che indica il pacchetto di mischia ed è una metafora del team di sviluppo che deve lavorare insieme in modo che tutti gli attori del progetto spingano nella stessa direzione, agendo come un’unica entità coordinata.


%**************************************************************
\section{L'offerta di Stage}
Da anni l'azienda collabora con l'università di Padova alla ricerca di neolaureati da formare ed inserire nel proprio organico e, grazie ad iniziative come StageIt le possibilità di incontro tra studenti ed aziende sono molto più concrete e produttive.\\ %TODO spiegare cos'è stageit?
\begin{figure}[h]
\centering
\includegraphics[height = 4 cm]{stageit2018}
\caption{Logo di StageIT}
\end{figure}\\ 
Gli studenti hanno infatti l' occasione di conoscere svariate realtà, anche piccole, che operano nel settore nei più disparati rami applicativi mentre per le aziende, oltre ad essere un'importante vetrina ed un momento di confronto con potenziali partner e concorrenti, costituisce anche la possibilità di conoscere tanti giovani studenti in un breve arco temporale.\\
\subsection{Il progetto}
Tra le varie proposte di stage che quest'azienda offriva, mi è stato proposto di collaborare a quello riguardante il loro software CRM.\\
Un software CRM, acronimo di Costumer Relationship Management, è un prodotto che permette all'utilizzatore di mantenere una rete di relazioni con clienti e potenziali clienti, non soltanto al fine di mantenere una relazione commerciale ma di fiducia reciproca che si possa protrarre nel tempo, anche attraverso l'impiego di strumenti di fidelizzazione come newsletters, campagne ed eventi. \\
In questo contesto si inserisce  lo stage cui ho preso parte e che aveva come obiettivo la ridefinizione della form principale di tale software, dapprima realizzato in \emph{Java} con traduzione in \emph{JavaScript} a cura della libreria \emph{GWT} ed ora richiesto in \emph{JavaScript} nativo, utilizzando una libreria da definire in fase di svolgimento dello stage dopo un'adeguata analisi delle alternative open-source disponibili sul mercato\footcite{queste ed altre tecnologie verranno discusse nel dettaglio nei capitoli seguenti}. 

%**************************************************************
\subsection{Gli obiettivi}
Gli obiettivi del progetto di stage sono elencati nella tabella 1.\\ %todo.
Essi sono classificati da:

\begin{itemize}
\item un ID, che rappresenta univocamente l'obiettivo;
\item un aggettivo di importanza dell'obiettivo;
\item una breve descrizione testuale.
\end{itemize}

L'ID è formato da una lettera iniziale (P o F), ad indicare se il requisito da soddisfare sia di tipo \emph{Produttivo} o \emph{Formativo}, e da un numero sequenziale.\\
L'agettivo di importanza può essere: \emph{Obbligatorio}, \emph{Desiderabile} oppure \emph{Facoltativo}, in base alla necessità che ha lo lo stesso di essere soddisfatto.\\
La breve descrizione testuale descrive nel modo più concreto e riassuntivo possibile l'obiettivo.\\

\begin{table}[h]
\centering
\caption{Obiettivi dello stage}
\label{tab:obiettivi}
\begin{tabular}{|l|c|p{7cm}|}
\hline
ID & Importanza & Descrizione \\
\hline
F1 & Obbligatorio & Acquisizione delle competenze di base sulla famiglia di software denominata CRM.\\
\hline
F2 & Obbligatorio & Acquisizione delle competenze di base sul software di sviluppo GWT.\\
\hline
F3 & Desiderabile & Acquisizione di competenze avanzate sul linguaggio di programmazione utilizzato per lo sviluppo del prototipo.\\
\hline
P1 & Obbligatorio & Analisi dei requisiti tecnici ed applicativi.\\
\hline
P2 & Obbligatorio & Analisi dell'User Interface.\\
\hline
P3 & Obbligatorio & Analisi degli Use Case.\\
\hline
P4 & Obbligatorio & Sviluppo di un prototipo che implementi le stesse funzionalità del componente esistente.\\
\hline
P5 & Facoltativo & Implementazione di nuove funzionalità di interazione con l'utente sul prototipo sviluppato.\\
\hline
\end{tabular}
\end{table}

%*************************************************************

\subsection{Pianificazione del lavoro}
In accordo con l'azienda, è stato redatto un piano di lavoro a granularità settimanale. \\
Il piano, mostrato in tabella, descrive per ogni settimana di stage il lavoro di approfondimento e sviluppo necessari al raggiungimento degli obiettivi. \\%todo link tabella \\

\textbf{Settimana 1:}\\
La prima settimana è stata volta all'introduzione nell'azienda, al concetto di CRM, all'installazione dell'ambiente di sviluppo ed ad una prima fase di formazione sul funzionamento del software JGalileo CRM.\\
\textbf{Settimane 2 e 3:}\\
Nella seconda settimane di stage ho approfondito la conoscenza del software JGalileo CRM e attuato l'analisi dei requisiti tecnici ed applicativi, che si è protratta fino alla fine della terza settimana.\\
\textbf{Settimane 4:}\\
La quarta settimana è stata dedicata all'analisi di varie alternative open-source per il successivo sviluppo del prototipo. \\
\textbf{Settimane 5-8:}\\
La seconda metà dello stage è stata interamente dedicata allo sviluppo del prototipo, utilizzando la tecnologia scelta durante il precedente periodo.\\

\begin{table}[h]
\centering
\caption{Pianificazione del lavoro}
\label{tab:pianificazione-del-lavoro}
\begin{tabular}{|l|l|c|p{7cm}|}
\hline
Periodo  & Dal & Al &Descrizione\\
\hline
Prima settimana  & 21-05 & 25-05 & Ricerca, studio e documentazione per inquadramento del
progetto.
Introduzione al concetto di CRM ed al software JGalileo CRM\\
\hline
Seconda settimana  & 28-05 & Al 1-06 & Analisi dei requisiti applicativi e tecnici\\
\hline
Terza settiamana & 04-06 & 08-06 &Analisi dei requisiti applicativi e tecnici\\
\hline
Quarta settiamana  & 11-06 & 15-06 & Ricerca degli ambienti di sviluppo open source soddisfacenti i requisiti richiesti e scelta dell’ambiente di sviluppo\\
\hline
Quinta settimana  & 18-06 & 22-06 & Sviluppo prototipo\\
\hline
Sesta settimana  & 25-06 & 29-06 & Sviluppo prototipo\\
\hline
Settima settimana  & 02-07 & 06-07 &Sviluppo prototipo\\
\hline
Ottava settimana  & 09-07 & 13-07 &Sviluppo prototipo\\
\hline
\end{tabular}
\end{table}

\section {Analisi dei rischi}
Ho voluto analizzare una serie di rischi che sarebbero potuti incorrere durante il periodo di stage. Sono riassunti nella tabella\\ %todo

\begin{table} %todo
\centering
\caption{Analisi preventiva dei rischi}
\label{tab:analisi-dei-rischi}
\begin{tabular}{|l|c|r|}
\hline
Descrizione & Trattamento & Rischio\\
\hline

\end{tabular}
\end{table}

\section {Obiettivi personali}
Nell'intraprendere il mio percorso di stage, oltre agli obiettivi stabiliti nel Piano di Lavoro, avevo anche degli obiettivi personali da raggiungere, di seguito elencati:
\begin{itemize}
\item Conoscere uno dei rami lavorativi in cui è possibile inserirsi al termine del percorso di studi;
\item Conoscere tecnologie nuove e non affrontate durante il percorso di studi;
\item Confrontarmi con software di larga scala;
\item Confrontarmi con altre persone già inserite nel mondo del lavoro;
\item Migliorarmi grazie all'aiuto del tem di lavoro;
\item Migliorarmi nella gestione di tempi/risorse nel contesto di un lavoro assegnatomi.
\end{itemize}