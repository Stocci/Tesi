% !TEX encoding = UTF-8
% !TEX TS-program = pdflatex
% !TEX root = ../tesi.tex

%**************************************************************
\chapter{Introduzione}
\label{cap:introduzione}
%**************************************************************

Introduzione al contesto applicativo.\\

\noindent Esempio di utilizzo di un termine nel glossario \\
\gls{api}. \\

\noindent Esempio di citazione in linea \\
\cite{site:agile-manifesto}. \\

\noindent Esempio di citazione nel pie' di pagina \\
citazione\footcite{womak:lean-thinking} \\

%**************************************************************
\section{L'azienda}

Sanmarco Informatica Spa è un'azienda italiana leader nella progettazione e realizzazione di soluzioni a supporto della riorganizzazione di tutti i processi aziendali e professionali. \\ Ha sede a  Grisignano di Zocco (VI), è attiva da 35 anni, può contare su più di 400 dipendenti e collaboratori, per seguire quotidianamente più di 1000 aziende clienti. \\ 
Il prodotto di punta di  Sanmarco Informatica è \emph{JGalileo}, un software gestionale \emph{ERP}\glsfirstoccur utilizzato, oltre che da aziende europee, anche in USA, Russia e Cina.\\

Nel centro di Sviluppo dell'azienda viene adottato il metodo Agile: esso nasce nel 2001 e definisce i dodici principi fondamentali per sviluppare con tempistiche e procedure ottimizzate un software che sia per il cliente una soluzione di successo. La metodologia “Agile” privilegia \emph{gli individui e le interazioni più che i processi e gli strumenti}, \emph{il software funzionante più che la documentazione esaustiva}, \emph{la collaborazione col cliente più che la negoziazione dei contratti}, \emph{rispondere al cambiamento più che seguire un piano}. \\ L’obiettivo è quello di garantire all’azienda tutta la flessibilità e la versatilità necessaria alla realizzazione di progetti efficaci. Questo metodo di lavoro si concentra su una collaborazione più face-to-face fra colleghi con cadenza regolare e ravvicinata, permettendo di mantenere e perfezionare man mano il focus sui progetti, valorizzando i nuovi punti di vista e le nuove risposte alle problematiche da affrontare. \\ In particolare, il metodo di gestione progetti “SCRUM” prevede di dividere il progetto in blocchi (Sprint), all’interno dei quali vengono sviluppate delle funzioni/moduli/applicazioni complete (denominate storie) pronte per la potenziale installazione al cliente. Il termine Scrum è mutuato dal termine del Rugby che indica il pacchetto di mischia ed è una metafora del team di sviluppo che deve lavorare insieme in modo che tutti gli attori del progetto spingano nella stessa direzione, agendo come un’unica entità coordinata.


%**************************************************************
\section{L'idea}

Introduzione all'idea dello stage.

%**************************************************************
\section{Organizzazione del testo}

\begin{description}
    \item[{\hyperref[cap:processi-metodologie]{Il secondo capitolo}}] descrive ...
    
    \item[{\hyperref[cap:descrizione-stage]{Il terzo capitolo}}] approfondisce ...
    
    \item[{\hyperref[cap:analisi-requisiti]{Il quarto capitolo}}] approfondisce ...
    
    \item[{\hyperref[cap:progettazione-codifica]{Il quinto capitolo}}] approfondisce ...
    
    \item[{\hyperref[cap:verifica-validazione]{Il sesto capitolo}}] approfondisce ...
    
    \item[{\hyperref[cap:conclusioni]{Nel settimo capitolo}}] descrive ...
\end{description}

Riguardo la stesura del testo, relativamente al documento sono state adottate le seguenti convenzioni tipografiche:
\begin{itemize}
	\item gli acronimi, le abbreviazioni e i termini ambigui o di uso non comune menzionati vengono definiti nel glossario, situato alla fine del presente documento;
	\item per la prima occorrenza dei termini riportati nel glossario viene utilizzata la seguente nomenclatura: \emph{parola}\glsfirstoccur;
	\item i termini in lingua straniera o facenti parti del gergo tecnico sono evidenziati con il carattere \emph{corsivo}.
\end{itemize}