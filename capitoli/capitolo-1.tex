% !TEX encoding = UTF-8
% !TEX TS-program = pdflatex
% !TEX root = ../tesi.tex

%**************************************************************
\chapter{Introduzione}
\label{cap:introduzione}
%**************************************************************

Introduzione al contesto applicativo.\\

\noindent Esempio di utilizzo di un termine nel glossario \\
\gls{api}. \\

\noindent Esempio di citazione in linea \\
\cite{site:agile-manifesto}. \\


%**************************************************************
\section{L'azienda}



\begin{figure}[h]
\centering
\includegraphics[height = 5 cm]{sanmarco-logo}
\caption{Logo di Sanmarco Informatica}
\end{figure}
Sanmarco Informatica Spa è un'azienda italiana leader nella progettazione e realizzazione di soluzioni a supporto della riorganizzazione di tutti i processi aziendali e professionali. \\ Ha sede a  Grisignano di Zocco (VI), è attiva da 35 anni, può contare su più di 400 dipendenti e collaboratori, per seguire quotidianamente più di 1000 aziende clienti. \\ 
Il punto di forza dell'azienda sta nella formazione del personale e nella ricerca, campi nei quali viene investito in media il 20\% del fatturato annuo. \\
Il prodotto di punta di  Sanmarco Informatica è \emph{JGalileo}, un software gestionale \emph{ERP} utilizzato, oltre che da aziende europee, anche in USA, Russia e Cina.\\
Un software \emph{ERP, Enterprise Resource Planning}, pianificazione delle risorse d'impresa, è un sistema di gestione che integra tutti i processi di business rilevanti di un'azienda (vendite, acquisti, gestione magazzino, contabilità ecc.).\\Esso si pone l'obiettivo di rendere fluido il passaggio di informazioni tra le varie divisioni di un'azienda cliente, ad esempio allo scopo di ottimizzare la produzione in base alla richiesta, senza produrre e poi stockare in magazzino.\\
Oltre a JGalileo, Sanmarco si occupa di svariati prodotti software, tutti potenzialmente interconnessi e che occupano variegati settori di mercato: dal commercio B2B e B2C, alla Business Intelligence al CRM, quest'ultimo oggetto del mio progetto di stage.\\ 

\subsection{Sede di lavoro}
Sanmarco Informatica, in Italia, può contare su 5 sedi: la sede principale ed il centro Ricerca e Sviluppo si trovano a Grisignano di Zocco (VI), ci sono poi una filiale a Vimercate (MB), una a Tavagnacco (UD), ed una a Reggio Emilia.\\
La sede del mio stage è stato il centro di Ricerca e Sviluppo.\\
Qui viene adottato il metodo di lavoro Agile: esso nasce nel 2001 e definisce i dodici principi fondamentali per sviluppare con tempistiche e procedure ottimizzate un software che sia per il cliente una soluzione di successo. La metodologia “Agile” privilegia
\begin{itemize}
	\item \emph{gli individui e le interazioni più che i processi e gli strumenti}; 
	\item \emph{il software funzionante più che la documentazione esaustiva}; 
	\item \emph{la collaborazione col cliente più che la negoziazione dei contratti}; 
	\item \emph{rispondere al cambiamento più che seguire un piano}. 
\end{itemize} 
	L’obiettivo è quello di garantire all’azienda tutta la flessibilità e la versatilità necessaria alla realizzazione di progetti efficaci. Questo metodo di lavoro si concentra su una collaborazione più face-to-face fra colleghi con cadenza regolare e ravvicinata, permettendo di mantenere e perfezionare man mano il focus sui progetti, valorizzando i nuovi punti di vista e le nuove risposte alle problematiche da affrontare. \\ In particolare, il metodo di gestione progetti “SCRUM” prevede di dividere il progetto in blocchi (Sprint), all’interno dei quali vengono sviluppate delle funzioni/moduli/applicazioni complete (denominate storie) pronte per la potenziale installazione al cliente. Il termine Scrum è mutuato dal termine del Rugby che indica il pacchetto di mischia ed è una metafora del team di sviluppo che deve lavorare insieme in modo che tutti gli attori del progetto spingano nella stessa direzione, agendo come un’unica entità coordinata.


%**************************************************************
\section{L'offerta di Stage}
Da anni l'azienda collabora con l'università di Padova alla ricerca di neolaureati da formare ed inserire nel proprio organico e, grazie ad iniziative come StageIt le possibilità di incontro tra studenti ed aziende sono molto più concrete e produttive.
StageIT è un'iniziativa che mira ad agevolare l'incontro tra le imprese e gli studenti che entreranno a breve in stage nel mondo del lavoro con specifico riferimento al settore ICT, favorendo un'occasione di conoscenza reciproca mediante colloqui individuali.
\begin{figure}[h]
\centering
\includegraphics[height = 4 cm]{stageit2018}
\caption{Logo di StageIT}
\end{figure}
Gli studenti hanno infatti l' occasione di conoscere svariate realtà, anche piccole, che operano nel settore nei più disparati rami applicativi mentre per le aziende, oltre ad essere un'importante vetrina ed un momento di confronto con potenziali partner e concorrenti, costituisce anche la possibilità di conoscere tanti giovani studenti in un breve arco temporale.\\
\subsection{Il progetto}
Tra le varie proposte di stage che quest'azienda offriva, mi è stato proposto di collaborare a quello riguardante il software CRM da loro sviluppato.\\
Un software CRM, acronimo di Costumer Relationship Management, è un prodotto che permette all'utilizzatore di mantenere una rete di relazioni con clienti e potenziali clienti, non soltanto al fine di stabilire una relazione commerciale, ma di fiducia reciproca che si possa protrarre nel tempo, anche attraverso l'impiego di strumenti di fidelizzazione quali newsletters, campagne ed eventi. \\
In questo contesto si inserisce  lo stage cui ho preso parte e che aveva come obiettivo la ridefinizione della form principale di tale software.\\ Una form è l'interfaccia di un'applicazione che consente ad un utente di compilare dati e di inviarli ad un server, tipicamente per l'immissione degli stessi nel sistema.\\
Questa form, dapprima realizzata utilizzando il linguaggio \emph{Java} e successivamente tradotta in \emph{JavaScript} grazie all'utilizzo della libreria \emph{GWT}, è ora richiesta in \emph{JavaScript} nativo. Una fase del progetto prevede infatti la ricerca e l'analisi di framework e librerie JavaScript \emph{open-source} disponibili sul mercato\footcite{queste ed altre tecnologie verranno discusse nel dettaglio nei capitoli seguenti}, per poi utilizzare la libreria scelta allo scopo di sviluppare un prototipo di form che abbia le stesse funzionalità di quella attualmente in uso, e possibilmente vada ad aggiungere nuove funzionalità come ad esempio la possibilità per un utente di inserire, in completa autonomia, nuovi campi dati personalizzati all'interno delle form.\\
\\
Le principali difficoltà nell'affrontare questo progetto sono legate principalmente al dover studiare e capire un software molto grande e già in uso da molti anni e che implementa diverse tecnologie che non avevo mai affrontato durante il corso di studi, come Liferay, GWT ed Hibernate.\\
Oltre a ciò conoscere ed imparare ad utilizzare abbastanza velocemente una libreria JavaScript può portare a perdere molto tempo in prove ed esercitazioni prima di poter maneggiare con sufficiente sicurezza gli strumenti messi a disponizione da essa.\\ 
In terzo luogo è necessario che il codice da me prodotto non influenzi le altre funzionalità presenti all'interno del software: questo da un lato pone delle semplificazioni nello sviluppo, come ad esempio il riutilizzo di servizi REST oppure dei parametri che vengono passati in fase di chiamata al costruttore, dall'altro limita però la creatività e costringe l'adozione di un flusso di creazione e gestione abbastanza similare a quello già in uso.\\
\\
La principale soluzione alle difficoltà appena descritte consiste nel dedicare buona parte dell'esperienza allo studio, in particolare allo studio del software e della form già esistente al fine di capire al meglio le modalità di interazione tra quest'ultima e il resto del sistema, in special modo riguardo i dati che vengono trasmessi ed inviati. In questo modo diventa semplice capire di che risorse si dispone e di come ottimizzarne l'utilizzo.\\


%**************************************************************

\subsection{Pianificazione del lavoro}
In accordo con l'azienda, è stato redatto un piano di lavoro a granularità settimanale. \\
Il piano, mostrato in tabella, descrive per ogni settimana di stage il lavoro di approfondimento e sviluppo necessari al raggiungimento degli obiettivi. \\%todo link tabella \\

\textbf{Settimana 1:}\\
La prima settimana è stata volta all'introduzione nell'azienda, al concetto di CRM, all'installazione dell'ambiente di sviluppo ed ad una prima fase di formazione sul funzionamento del software JGalileo CRM.\\
\textbf{Settimane 2 e 3:}\\
Nella seconda settimane di stage ho approfondito la conoscenza del software JGalileo CRM e attuato l'analisi dei requisiti tecnici ed applicativi, che si è protratta fino alla fine della terza settimana.\\
\textbf{Settimane 4:}\\
La quarta settimana è stata dedicata all'analisi di varie alternative open-source per il successivo sviluppo del prototipo. \\
\textbf{Settimane 5-8:}\\
La seconda metà dello stage è stata interamente dedicata allo sviluppo del prototipo, utilizzando la tecnologia scelta durante il precedente periodo.\\

\begin{table}[h]
\centering
\caption{Pianificazione del lavoro}
\label{tab:pianificazione-del-lavoro}
\begin{tabular}{|l|l|c|p{7cm}|}
\hline
Periodo  & Dal & Al &Descrizione\\ %TODO provare a rendere mene granulare la tabella
\hline
Prima settimana  & 21-05 & 25-05 & Ricerca, studio e documentazione per inquadramento del
progetto.
Introduzione al concetto di CRM ed al software JGalileo CRM\\
\hline
Seconda settimana  & 28-05 & 01-06 & Analisi dei requisiti applicativi e tecnici\\
\hline
Terza settiamana & 04-06 & 08-06 &Analisi dei requisiti applicativi e tecnici\\
\hline
Quarta settiamana  & 11-06 & 15-06 & Ricerca degli ambienti di sviluppo open source soddisfacenti i requisiti richiesti e scelta dell’ambiente di sviluppo\\
\hline
Quinta settimana  & 18-06 & 22-06 & Sviluppo prototipo\\
\hline
Sesta settimana  & 25-06 & 29-06 & Sviluppo prototipo\\
\hline
Settima settimana  & 02-07 & 06-07 &Sviluppo prototipo\\
\hline
Ottava settimana  & 09-07 & 13-07 &Sviluppo prototipo\\
\hline
\end{tabular}
\end{table}

\section{Struttura del testo}
I capitoli seguenti descriveranno in maniera approfondita strumenti e metodologie con le quali è stato affrontato il periodo di stage.\\
In particolare, secondo capitolo enuncia le tecnologie con le quali ho dovuto interfacciarmi durante lo studio del progetto o durante la fase di sviluppo.\\
Il terzo capitolo elenca i requisiti ed i casi d'uso del progetto.\\
Il quarto capitolo descrive lo stato dell'applicazione esistente, con particolare riguardo alla parte di mio interesse.\\
Il quinto capitolo illustra lo sviluppo del protoptipo oggetto del mio progetto, iniziando dallo studio delle tecnologie utilizzate.\\
\newpage