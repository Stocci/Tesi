% !TEX encoding = UTF-8
% !TEX TS-program = pdflatex
% !TEX root = ../tesi.tex

%**************************************************************
\chapter{Processi e metodologie}
\label{cap:processi-metodologie}
%**************************************************************

\intro{In questo capitolo verranno descritte le principali tecnologie che ho approfondito durante l'esperienza di stage.}\\

%**************************************************************
\section{Tecnologie utilizzate}
Durante la mia esperienza di stage ho potuto prendere contatto con molte tecnologie differenti e che non avevo incontrato durante il corso di studi. Esse spaziano dall'ambiente server a quello client. L'approfondimento di tali tecnologie, specialmente nelle prime settimane di stage, è stato fondamentale per capire il funzionamento del software da modificare \\
\subsection{Java}
\begin{figure}[h]
	\centering
	\includegraphics[height = 4 cm]{java-logo}
	\caption{Logo di Java}
\end{figure}
Java è uno dei linguaggi di programmazione general-purpose più popolari al mondo. \\
Rilasciato nel 1995, riprende molta della sintassi dal linguaggio C e C++, ma spostando parte delle responsabilità prima lasciate ai programmatori, come la gestione della memoria, alla \emph{Java Virtual Machine (JVM)}: una macchina virtuale su chi viene eseguito tutto il codice Java.\\
La presenza di una macchina virtuale crea un livello aggiuntivo tra il sistema operativo e l' \emph{IDE} di sviluppo: Questo permette al linguaggio Java di aderire al principio \emph{Write Once Run Anywhere (WORA)}: Il codice compilato (bytecode) può essere eseguito da qualsisasi computer provvisto di una JVM, senza necessitare di ricompilazioni e perfino adattamenti del codice sorgente in base al sistema operativo presente sul computer.
Java è inoltre molto utilizzato per eseguire applicazioni client-server, cioè applicazioni (Client) che per ottentere i dati necessari si appoggiano ad un fornitore (Server), che recupera per loro i dati necessari, in maniera del tutto trasparente al client e quindi all'utente.\\
Java dispone di una libreria proprietaria specializzata, ma in JGalileoCRM tale libreria è sostituita da \emph{Liferay}.\\ %TODO occhio che java.net è chiusa
\subsection{Liferay}
\begin{figure}[h]
	\centering
	\includegraphics[height = 4 cm]{Logo-Liferay}
	\caption{Logo di Liferay}
\end{figure}
Liferay è una tecnologia \emph{enterprise portal} open source, realizzato in Java.\\
Per enterprise portal si intende un sistema informatico evoluto, in grado di integrare informazioni, processi e persone allo scopo di fornire valore aggiunto in termini di:
\begin{itemize}
	\item Gestione del \emph{Single Sign On};
	\item Semplice personalizzazioni ad hoc per ogni cliente;
	\item Analisi delle pagine, in termini di accessi, click, download molto semplice;
	\item Integrazione tra funzionalità e dati di diversi sistemi in nuove componenti definite portlets. 
\end{itemize}
Le portlets sono il cuore del sistema di Liferay. Esse infatti permettono di concentrare lo sviluppo solamente sulla gestione della funzionalità principale, lasciando a liferay la gestione degli accessi, dei menù di navigazione e degli altri componenti globali dell'applicazione.\\

\subsection{GWT}
\begin{figure}[h]
	\centering
	\includegraphics[height = 4 cm]{gwt-logo}
	\caption{Logo di Google Web Toolkit}
\end{figure} 
GWT, acronimo di \emph{Google Web Toolkit} è un insieme di tool open source che permette la creazione ed il mantenimento di complesse applicazioni front-end JavaScript scritte in Java.\\
GWT infatti si occupa di tradurre il codice scritto in Java in codice JavaScript, interpretabile da tutti i moderni browsers Tutto il codice Java può essere compilato ed eseguito grazie ai file Ant inclusi.
Ant è un progetto open source della Apache foundation, volto ad automatizzare il processo di build. è simile al comando make di Unix, ma scritto in Java. \\ %TODO make come codice
Il plug-in di Google per Eclipse (IDE in uso presso Sanmarco Informatica) è molto comleto, offrendo la possibilità di creare progetti, farne il debug, il testing, meccanismi di validazione e di controllo della sintassi.\\
\subsection{Apache Tomcat}
\begin{figure}[h]
	\centering
	\includegraphics[height = 4 cm]{apache_tomcat-card}
	\caption{Logo di Apache Tomcat}
\end{figure}
Apache Tomcat è un web server, cioè un' applicazione web che, in esecuzione su di un server, gestisce le richieste di trasferimento tra le varie pagine web di un client.\\
In particolare, Tomcat opera utilizzando servlet, cioè oggetti scritti in Java molto usati nella generazione di pagine web dinamiche.
Queste servlet, in JGalileoCRM non vengono direttamente scritte in Java, ma sono scritte con \emph{JavaServer Pages (JSP)}:\\
JSP è una tecnologia di programmazione web, scritta in Java, che si basa sull'uso di speciali tag all'interno di una pagina HTML, con cui possono essere chiamate funzioni scritte in linguaggio Java o JavaScript.
A runtime, le pagine JSP vengono tradotte automaticamente in servlet utilizzabili da Tomcat. 
%TODO vantaggi di jsp
\subsection{Database SQL}
\subsection{Database AS400}
\subsection{Hibernate}
\subsection{Javascript}
\subsubsection{Analisi delle alternative}
\subsubsection{React}
\subsubsection{Angular}
\subsubsection{Vue.JS}
\subsubsection{Scelta finale}