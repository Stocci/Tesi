% !TEX encoding = UTF-8
% !TEX TS-program = pdflatex
% !TEX root = ../tesi.tex

%**************************************************************
\chapter{Sviluppo del prototipo}
\label{cap:sviluppo-prototipo}
%**************************************************************

\intro{Questo capitolo illustra il funzionamento delle tecnologie che ho utilizzato nello sviluppo del prototipo, la logica di realizzazione ed il risultato ottenuto}\\

%**************************************************************
\section{AngularJS}
\label{sec:AngularJS}

La libreria che ho scelto di utilizzare, AngularJS, nasce nel 2012 nei laboratori di Google.\\
Esso potenzia l'aspetto dichiarativo del linguaggio \gls{html}, offrendo al contempo tutti gli strumenti necessari alla realizzazione di \gls{spag}. 
\subsection{Controller e direttive}
Una delle peculiarità di  AngularJS è che consente di aggiungere al codice HTML nuovi attributi, denominati \textbf{direttive}.\\
Tali direttive possono essere definite dall'utente, che comunque ha a disposizione anche un buon numero di 
direttive standard fornite dal \emph{framework}. Tali direttive consentono al compilatore HTML di AngularJS (attivato dal comando \lstinline{$scompile()})
\section{Codifica}





\newpage
