% !TEX encoding = UTF-8
% !TEX TS-program = pdflatex
% !TEX root = ../tesi.tex

%**************************************************************
\chapter{Approfondimenti}
%**************************************************************

\section{Funzionamento di Hibernate} 
\label{sec:appendice-1}
Come avviene la conversione delle tabelle in classi Java?\\
\begin{figure}[h]
	\centering
	\includegraphics[height = 5 cm]{hibernate_works}
	\caption{Schema del funzionamento di Hibernate}
	\label{schema-generale-hibernate}
\end{figure}
Hibernate deve conoscere la configurazione del database e per farlo necessita di un file, estesi generalmente in \textbf{.cfg.xml}. Al framework è inoltre indispensabile fornire dei files di mapping, uno per ogni tabella e generalmente estesi con i suffissi \textbf{.hmb.xml}, che contengono le informazioni riguardanti le colonne della singola tabella da trasformare in un oggetto Java.\\

\begin{figure}[h]
	\centering
	\includegraphics[height = 5 cm]{hibernate-entities}
	\caption{XML di un'entità Hibernate di JGalileo CRM}
	\label{entità}
\end{figure}

Questi files vengono poi utilizzati per creare una \emph{SessionFactory} globale e thread-safe che funge da \emph{gateway} per l'interrogazione del database. \\ I vantaggi dell'utilizzo della tecnica di programmazione ORM (Object-Relational Mapping) sono molteplici ed in particolare consentono:
\begin{itemize}
	\item \textbf{Disaccoppiamento dal DBMS};
	\item \textbf{Elevata portabilità};
	\item \textbf{Drastica riduzione del codice sorgente}, a causa dei semplici comandi che mascherano complesse istruzioni;
	\item \textbf{Elevata modularità}.
\end{itemize}

\section{Model View Controller}
\label{sec:MVC}
Il pattern \emph{MVC, Model-View-Controller}, è un design pattern, cioè uno schema di progettazione che costituisce una soluzione progettuale ad un problema ricorrente.\\
\'E uno dei pattern più famosi ed utilizzati per la separazione tra la logica di presentazione e la logica di business nei sistemi software, ed in particolare nelle applicazioni web.\\
\begin{figure}[h]
	\centering
	\includegraphics[height = 7 cm]{MVC}
	\caption{Design pattern MVC}
	\label{mvc-schema}
\end{figure}
Come evidenziato dalla figura \ref{mvc-schema}, vi sono 3 attori principali:
\begin{itemize}
	\item \textbf{Model}: rappresenta la cosiddetta \emph{logica di business}, cioè l'insieme di dati e metodi per eseguire operazioni;
	\item \textbf{View}rappresenta come i dati vengono visualizzati nell'interfaccia utente, cioè la \emph{logica di presentazione};
	\item \textbf{Controller}:si pone come intermediario tra view e model;
\end{itemize}
Il controller, che osserva la view, riceve da quest'ultima le richieste di elaborazione e, una volta che il model restituisce i dati, si occupa di inviarli alla view.\\
Questo design pattern ha molti lati positivi, tra i quali la divisione dei compiti ed il riuso di codice.\\


\newpage