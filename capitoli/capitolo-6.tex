% !TEX encoding = UTF-8
% !TEX TS-program = pdflatex
% !TEX root = ../tesi.tex

%**************************************************************
\chapter{Conclusioni}
\label{cap:conclusioni}
%**************************************************************

\intro{}

\section{Raggiungimento degli obiettivi}
La seguente tabella riassume il numero di requisiti obbligatori, desiderabili ed opzionali completati e non

\begin{table} %todo
	\centering
	\caption{Riepilogo del soffisfacimento rei requisiti}
	\label{tab:obiettivi-riepilogo}
	\begin{tabular}{|p{2,5cm}|p{2,5cm}|p{2,5cm}|p{2,5cm}|p{2,5cm}|}
		\hline
		\textbf{Tipo} & \textbf{Obbligatorio} & \textbf{Facoltativo} & \textbf{Desiderabile} & \textbf{Soddisfatti}\\
		\hline
		\textbf{Funzionali} & \textbf{59} & \textbf{0} & \textbf{2} & \textbf{60}\\
		\hline
		\textbf{Qualità} & \textbf{1} & \textbf{0} & \textbf{1} & \textbf{1}\\
		\hline
		\textbf{Vincolo} & \textbf{2} & \textbf{0} & \textbf{0} & \textbf{2}\\
		\hline
		\textbf{Totale} & \textbf{62} & \textbf{0} & \textbf{3} & \textbf{63}\\
		\hline	
	\end{tabular}
\end{table}

\section {Obiettivi personali}
Nell'intraprendere il mio percorso di stage, oltre agli obiettivi stabiliti nel Piano di Lavoro, avevo anche degli obiettivi personali da raggiungere, di seguito elencati:
\begin{itemize}
	\item Conoscere uno dei rami lavorativi in cui è possibile inserirsi al termine del percorso di studi;
	\item Conoscere tecnologie nuove e non affrontate durante il percorso di studi;
	\item Confrontarmi con software di larga scala;
	\item Confrontarmi con altre persone già inserite nel mondo del lavoro;
	\item Migliorarmi grazie all'aiuto del tem di lavoro;
	\item Migliorarmi nella gestione di tempi/risorse nel contesto di un lavoro assegnatomi.
\end{itemize}