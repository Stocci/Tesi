%**************************************************************
% file contenente le impostazioni della tesi
%**************************************************************

%**************************************************************
% Frontespizio
%**************************************************************

% Autore
\newcommand{\myName}{Davide Stocco}                                    
\newcommand{\myTitle}{Revisione dell'interfaccia grafica di un software CRM con l'uso di AngularJS}

% Tipo di tesi                   
\newcommand{\myDegree}{Tesi di laurea triennale}

% Università             
\newcommand{\myUni}{Università degli Studi di Padova}

% Facoltà       
\newcommand{\myFaculty}{Corso di Laurea in Informatica}

% Dipartimento
\newcommand{\myDepartment}{Dipartimento di Matematica "Tullio Levi-Civita"}

% Titolo del relatore
\newcommand{\profTitle}{Prof.}

% Relatore
\newcommand{\myProf}{Paolo Baldan}

% Luogo
\newcommand{\myLocation}{Padova}

% Anno accademico
\newcommand{\myAA}{2017-2018}

% Data discussione
\newcommand{\myTime}{Settembre 2018}


%**************************************************************
% Impostazioni di impaginazione
% see: http://wwwcdf.pd.infn.it/AppuntiLinux/a2547.htm
%**************************************************************

\setlength{\parindent}{14pt}   % larghezza rientro della prima riga
\setlength{\parskip}{0pt}   % distanza tra i paragrafi


%**************************************************************
% Impostazioni di biblatex
%**************************************************************
\bibliography{bibliografia} % database di biblatex 

\defbibheading{bibliography} {
    \cleardoublepage
    \phantomsection 
    \addcontentsline{toc}{chapter}{\bibname}
    \chapter*{\bibname\markboth{\bibname}{\bibname}}
}

\setlength\bibitemsep{1.5\itemsep} % spazio tra entry

\DeclareBibliographyCategory{opere}
\DeclareBibliographyCategory{web}

\addtocategory{opere}{womak:lean-thinking}
\addtocategory{web}{site:agile-manifesto}

\defbibheading{opere}{\section*{Riferimenti bibliografici}}
\defbibheading{web}{\section*{Siti Web consultati}}


%**************************************************************
% Impostazioni di caption
%**************************************************************
\captionsetup{
    tableposition=top,
    figureposition=bottom,
    font=small,
    format=hang,
    labelfont=bf
}

%**************************************************************
% Impostazioni di glossaries
%**************************************************************

%**************************************************************
% Acronimi
%**************************************************************
\renewcommand{\acronymname}{Acronimi e abbreviazioni}

\newacronym[description={\glslink{apig}{Application Program Interface}}]
    {api}{API}{Application Program Interface}

\newacronym[description={\glslink{umlg}{Unified Modeling Language}}]
    {uml}{UML}{Unified Modeling Language}

\newacronym[description={\glslink{sqlg}{Strucured Query Language}}]
    {sql}{SQL}{Strucured Query Language}
    
\newacronym[description={\glslink{domg}{Document Object Model}}]
	{dom}{DOM}{Document Object Model}
	
\newacronym[description={\glslink{spag}{Single Page Applicaion}}]
{spa}{SPA}{Single Page Application}

\newacronym[description={\glslink{phpg}{PHP: Hypertext Preprocessor}}]
{php}{PHP}{PHP: Hypertext Preprocessor}

\newacronym[description={\glslink{urlg}{Uniform Resource Locator}}]
{url}{URL}{Uniform Resource Locator}

\newacronym[description={\glslink{html}{HypetText Markup Language}}]
{html}{HTML}{HypetText Markup Language}
%**************************************************************
% Glossario
%**************************************************************
%\renewcommand{\glossaryname}{Glossario}

\newglossaryentry{apig}
{
    name=\glslink{api}{API},
    text=Application Program Interface,
    sort=api,
    description={in informatica con il termine \emph{Application Programming Interface API} (ing. interfaccia di programmazione di un'applicazione) si indica ogni insieme di procedure disponibili al programmatore, di solito raggruppate a formare un set di strumenti specifici per l'espletamento di un determinato compito all'interno di un certo programma. La finalità è ottenere un'astrazione, di solito tra l'hardware e il programmatore o tra software a basso e quello ad alto livello semplificando così il lavoro di programmazione}
}

\newglossaryentry{umlg}
{
    name=\glslink{uml}{UML},
    text=UML,
    sort=uml,
    description={in ingegneria del software \emph{UML, Unified Modeling Language} (ing. linguaggio di modellazione unificato) è un linguaggio di modellazione e specifica basato sul paradigma object-oriented. L'\emph{UML} svolge un'importantissima funzione di ``lingua franca'' nella comunità della progettazione e programmazione a oggetti. Gran parte della letteratura di settore usa tale linguaggio per descrivere soluzioni analitiche e progettuali in modo sintetico e comprensibile a un vasto pubblico}
}

\newglossaryentry{phpg}
{
	name=\glslink{php}{PHP},
	text=PHP,
	sort=php,
	description={l'acronimo ricorsivo \emph{PHP, PHP: Hypertext Preprocessor} (ing. PHP: preprocessore di ipertesti) è un linguaggio open-source di scripting concepito per la programmazione di pagine web dinamiche, anche se attualmente il suo uso più comune risiede nelle applciazioni web lato server.}
}

\newglossaryentry{sqlg}
{
	name=\glslink{sql}{SQL},
	text=SQL,
	sort=sql,
	description={in informatica il termine \emph{SQL, Structured Query Language} è un linguaggio standard per l'interrogazione di database basati sul modello realzionale, cioè dove i dati sono inseriti in tabelle come valori di attributi e messi in relazione tra di loro }
}

\newglossaryentry{domg}
{
	name=\glslink{dom}{DOM},
	text=DOM,
	sort=dom,
	description={l'acronimo \emph{DOM, Document Object Model} (ing.  modello ad oggetti del documento) è una forma di rappresentazione di documenti, rappresentato come modello orientato agli oggetti.\\
	Tipicamente, nella rappresentazione di un documento HTML, questo modello corrisponde ad un albero dove la radice è l'elemento più generale del documento (il tag <html>), e le foglie i tag più annidati.}
}

\newglossaryentry{spag}
{
	name=\glslink{spa}{SPA},
	text=SPA,
	sort=spa,
	description={in informatica, per \emph{SPA, Single Page Application} (ing. applicazioni a pagina singola) s'intende un'applicazione web che sia, a livello di esperienza utente, più simili alle applicazioni desktop. Infatti in una \emph{SPA} il codice necessario viene caricato dinamicamente all'occorrenza, in questo modo la pagina non si ricaricherà in nessun punto del del processo. Spesso si rende necessaria una comunicazione dinamica con il web server.}
}


\newglossaryentry{script}
{
	name=\glslink{script}{SCRIPT},
	text=script,
	sort=script,
	description={uno \emph{script}, nel linguaggio informatico, è un particolare tipo di programma. Si differenzia infatti dai normali programmi da questi fattori:
	\begin{itemize}
		\item Bassa complessità;
		\item Uso di un linguaggio interpretato;
		\item Mancanza di un'interfaccia grafica.
	\end{itemize}
	Solitamente uno script è quindi una piccola funzionalità che risolve un prolblema specifico, inserita all'interno di un contesto più grande.}
}

\newglossaryentry{portlet}
{
	name=\glslink{portlet}{PORTLET},
	text=portlet,
	sort=portlet,
	description={una portlet è un modulo web riutilizzabile all'interno di portale web. Solitamente, infatti, una pagina di un portale è costituito da finestre il cui contenuto è diverso a seconda della portlet che andiamo ad inserire. Ad esempio possono esserci portlet per le previsioni meteo, per la geolocalizzazione, per l'inserimento di dati,.. \\
	Queste portlet, in quanto finestre, sono adattabili alle esigenze del singolo utente e possono quindi esse chiuse, allargate/ridotte, spostate.}
}
\newglossaryentry{middleware}
{
	name=\glslink{middleware}{MIDDLEWARE},
	text=middleware,
	sort=middleware,
	description={con il termine \emph{middleware} si intende un insieme di programmi informatici che fungono da intermediari tra diverse apllicazioni e componenti software.}
}

\newglossaryentry{urlg}
{name=\glslink{url}{URL},
text=URL,
sort=url,
description={un URL, Uniform Resource Locator, è l'indirizzo univoco in cui si trova una risorsa, come ad esempio un documento, immagine o video all'interno della rete internet.}
}

\newglossaryentry{record}
{name=\glslink{record}{RECORD},
	text=record,
	sort=record,
	description={Nel contesto corrente, per record si intende l'insieme dei dati inseriti da un utente, nel caso di una form per l'inserimento di una nuova scheda, oppure dell' insieme dati, recuperati dal database, riguardanti una particolare scheda.}
}

\newglossaryentry{htmlg}
{name=\glslink{html}{HTML},
	text=HTML,
	sort=html,
	description={HTML, HyperText Markup Language, è il principale linguaggio utilizzato per rappresentare le pagine web nella rete internet. è definito come linguaggio di markup, nel senso che definisce, tamiti elementi definiti tag, le modalità con cui un browser deve rappresentare la porzione di testo racchiusa all'interno del tag}
} % database di termini
\makeglossaries


%**************************************************************
% Impostazioni di graphicx
%**************************************************************
\graphicspath{{immagini/}} % cartella dove sono riposte le immagini


%**************************************************************
% Impostazioni di hyperref
%**************************************************************
\hypersetup{
    %hyperfootnotes=false,
    %pdfpagelabels,
    %draft,	% = elimina tutti i link (utile per stampe in bianco e nero)
    colorlinks=true,
    linktocpage=true,
    pdfstartpage=1,
    pdfstartview=FitV,
    % decommenta la riga seguente per avere link in nero (per esempio per la stampa in bianco e nero)
    %colorlinks=false, linktocpage=false, pdfborder={0 0 0}, pdfstartpage=1, pdfstartview=FitV,
    breaklinks=true,
    pdfpagemode=UseNone,
    pageanchor=true,
    pdfpagemode=UseOutlines,
    plainpages=false,
    bookmarksnumbered,
    bookmarksopen=true,
    bookmarksopenlevel=1,
    hypertexnames=true,
    pdfhighlight=/O,
    %nesting=true,
    %frenchlinks,
    urlcolor=webbrown,
    linkcolor=RoyalBlue,
    citecolor=webgreen,
    %pagecolor=RoyalBlue,
    %urlcolor=Black, linkcolor=Black, citecolor=Black, %pagecolor=Black,
    pdftitle={\myTitle},
    pdfauthor={\textcopyright\ \myName, \myUni, \myFaculty},
    pdfsubject={},
    pdfkeywords={},
    pdfcreator={pdfLaTeX},
    pdfproducer={LaTeX}
}

%**************************************************************
% Impostazioni di itemize
%**************************************************************
\renewcommand{\labelitemi}{$\ast$}

%\renewcommand{\labelitemi}{$\bullet$}
%\renewcommand{\labelitemii}{$\cdot$}
%\renewcommand{\labelitemiii}{$\diamond$}
%\renewcommand{\labelitemiv}{$\ast$}


%**************************************************************
% Impostazioni di listings
%**************************************************************
\lstset{
    language=[LaTeX]Tex,%C++,
    keywordstyle=\color{RoyalBlue}, %\bfseries,
    basicstyle=\small\ttfamily,
    %identifierstyle=\color{NavyBlue},
    commentstyle=\color{Green}\ttfamily,
    stringstyle=\rmfamily,
    numbers=none, %left,%
    numberstyle=\scriptsize, %\tiny
    stepnumber=5,
    numbersep=8pt,
    showstringspaces=false,
    breaklines=true,
    frameround=ftff,
    frame=single
} 


%**************************************************************
% Impostazioni di xcolor
%**************************************************************
\definecolor{webgreen}{rgb}{0,.5,0}
\definecolor{webbrown}{rgb}{.6,0,0}


%**************************************************************
% Altro
%**************************************************************

\newcommand{\omissis}{[\dots\negthinspace]} % produce [...]

% eccezioni all'algoritmo di sillabazione
\hyphenation
{
    ma-cro-istru-zio-ne
    gi-ral-din
}

\newcommand{\sectionname}{sezione}
\addto\captionsitalian{\renewcommand{\figurename}{Figura}
                       \renewcommand{\tablename}{Tabella}}

\newcommand{\glsfirstoccur}{\ap{{[g]}}}

\newcommand{\intro}[1]{\emph{\textsf{#1}}}

%**************************************************************
% Environment per ``rischi''
%**************************************************************
\newcounter{riskcounter}                % define a counter
\setcounter{riskcounter}{0}             % set the counter to some initial value

%%%% Parameters
% #1: Title
\newenvironment{risk}[1]{
    \refstepcounter{riskcounter}        % increment counter
    \par \noindent                      % start new paragraph
    \textbf{\arabic{riskcounter}. #1}   % display the title before the 
                                        % content of the environment is displayed 
}{
    \par\medskip
}

\newcommand{\riskname}{Rischio}

\newcommand{\riskdescription}[1]{\textbf{\\Descrizione:} #1.}

\newcommand{\risksolution}[1]{\textbf{\\Soluzione:} #1.}

%**************************************************************
% Environment per ``use case''
%**************************************************************
\newcounter{usecasecounter}             % define a counter
\setcounter{usecasecounter}{0}          % set the counter to some initial value

%%%% Parameters
% #1: ID
% #2: Nome
\newenvironment{usecase}[2]{
    \renewcommand{\theusecasecounter}{\usecasename #1}  % this is where the display of 
                                                        % the counter is overwritten/modified
    \refstepcounter{usecasecounter}             % increment counter
    \vspace{10pt}
    \par \noindent                              % start new paragraph
    {\large \textbf{\usecasename #1: #2}}       % display the title before the 
                                                % content of the environment is displayed 
    \medskip
}{
    \medskip
}

\newcommand{\usecasename}{UC}

\newcommand{\usecaseactors}[1]{\textbf{\\Attori Principali:} #1. \vspace{4pt}}
\newcommand{\usecasepre}[1]{\textbf{\\Precondizioni:} #1. \vspace{4pt}}
\newcommand{\usecasedesc}[1]{\textbf{\\Descrizione:} #1. \vspace{4pt}}
\newcommand{\usecasepost}[1]{\textbf{\\Postcondizioni:} #1. \vspace{4pt}}
\newcommand{\usecasealt}[1]{\textbf{\\Scenario Alternativo:} #1. \vspace{4pt}}

%**************************************************************
% Environment per ``namespace description''
%**************************************************************

\newenvironment{namespacedesc}{
    \vspace{10pt}
    \par \noindent                              % start new paragraph
    \begin{description} 
}{
    \end{description}
    \medskip
}

\newcommand{\classdesc}[2]{\item[\textbf{#1:}] #2}