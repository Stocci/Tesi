
%**************************************************************
% Acronimi
%**************************************************************
\renewcommand{\acronymname}{Acronimi e abbreviazioni}

\newacronym[description={\glslink{apig}{Application Program Interface}}]
    {api}{API}{Application Program Interface}

\newacronym[description={\glslink{umlg}{Unified Modeling Language}}]
    {uml}{UML}{Unified Modeling Language}

\newacronym[description={\glslink{sqlg}{Strucured Query Language}}]
    {sql}{SQL}{Strucured Query Language}
    
\newacronym[description={\glslink{domg}{Document Object Model}}]
	{dom}{DOM}{Document Object Model}
	
\newacronym[description={\glslink{spag}{Single Page Applicaion}}]
{spa}{SPA}{Single Page Application}

\newacronym[description={\glslink{phpg}{PHP: Hypertext Preprocessor}}]
{php}{PHP}{PHP: Hypertext Preprocessor}

\newacronym[description={\glslink{urlg}{Uniform Resource Locator}}]
{url}{URL}{Uniform Resource Locator}

\newacronym[description={\glslink{html}{HypetText Markup Language}}]
{html}{HTML}{HypetText Markup Language}

\newacronym[description={\glslink{ajax}{Asyncronous JavaScript And XML}}]
{ajax}{AJAX}{Asyncronous JavaScript And XML}

\newacronym[description={\glslink{ict}{Information and Communication Technologies}}]
{ict}{ICT}{Information and Communication Technologies}

\newacronym[description={\glslink{rest}{REpresentational State Transfert}}]
{rest}{REST}{REpresentational State Transfert}
%**************************************************************
% Glossario
%**************************************************************
%\renewcommand{\glossaryname}{Glossario}

\newglossaryentry{apig}
{
    name=\glslink{api}{API},
    text=Application Program Interface,
    sort=api,
    description={in informatica con il termine \emph{Application Programming Interface API} (ing. interfaccia di programmazione di un'applicazione) si indica ogni insieme di procedure disponibili al programmatore, di solito raggruppate a formare un set di strumenti specifici per l'espletamento di un determinato compito all'interno di un certo programma. La finalità è ottenere un'astrazione, di solito tra l'hardware e il programmatore o tra software a basso e quello ad alto livello semplificando così il lavoro di programmazione}
}

\newglossaryentry{umlg}
{
    name=\glslink{uml}{UML},
    text=UML,
    sort=uml,
    description={in ingegneria del software \emph{UML, Unified Modeling Language} (ing. linguaggio di modellazione unificato) è un linguaggio di modellazione e specifica basato sul paradigma object-oriented. L'\emph{UML} svolge un'importantissima funzione di ``lingua franca'' nella comunità della progettazione e programmazione a oggetti. Gran parte della letteratura di settore usa tale linguaggio per descrivere soluzioni analitiche e progettuali in modo sintetico e comprensibile a un vasto pubblico}
}

\newglossaryentry{phpg}
{
	name=\glslink{php}{PHP},
	text=PHP,
	sort=php,
	description={l'acronimo ricorsivo \emph{PHP, PHP: Hypertext Preprocessor} (ing. PHP: preprocessore di ipertesti) è un linguaggio open-source di scripting concepito per la programmazione di pagine web dinamiche, anche se attualmente il suo uso più comune risiede nelle applciazioni web lato server}
}

\newglossaryentry{sqlg}
{
	name=\glslink{sql}{SQL},
	text=SQL,
	sort=sql,
	description={in informatica il termine \emph{SQL, Structured Query Language} è un linguaggio standard per l'interrogazione di database basati sul modello realzionale, cioè dove i dati sono inseriti in tabelle come valori di attributi e messi in relazione tra di loro}
}

\newglossaryentry{domg}
{
	name=\glslink{dom}{DOM},
	text=DOM,
	sort=dom,
	description={l'acronimo \emph{DOM, Document Object Model} (ing.  modello ad oggetti del documento) è una forma di rappresentazione di documenti, rappresentato come modello orientato agli oggetti.\\
	Tipicamente, nella rappresentazione di un documento HTML, questo modello corrisponde ad un albero dove la radice è l'elemento più generale del documento (il tag <html>), e le foglie i tag più annidati}
}

\newglossaryentry{spag}
{
	name=\glslink{spa}{SPA},
	text=SPA,
	sort=spa,
	description={in informatica, per \emph{SPA, Single Page Application} (ing. applicazioni a pagina singola) s'intende un'applicazione web che sia, a livello di esperienza utente, più simili alle applicazioni desktop. Infatti in una \emph{SPA} il codice necessario viene caricato dinamicamente all'occorrenza, in questo modo la pagina non si ricaricherà in nessun punto del del processo. Spesso si rende necessaria una comunicazione dinamica con il web server}
}


\newglossaryentry{script}
{
	name=\glslink{script}{SCRIPT},
	text=script,
	sort=script,
	description={uno \emph{script}, nel linguaggio informatico, è un particolare tipo di programma. Si differenzia infatti dai normali programmi da questi fattori:
	\begin{itemize}
		\item Bassa complessità;
		\item Uso di un linguaggio interpretato;
		\item Mancanza di un'interfaccia grafica.
	\end{itemize}
	Solitamente uno script è quindi una piccola funzionalità che risolve un prolblema specifico, inserita all'interno di un contesto più grande}
}

\newglossaryentry{portlet}
{
	name=\glslink{portlet}{PORTLET},
	text=portlet,
	sort=portlet,
	description={una portlet è un modulo web riutilizzabile all'interno di portale web. Solitamente, infatti, una pagina di un portale è costituito da finestre il cui contenuto è diverso a seconda della portlet che andiamo ad inserire. Ad esempio possono esserci portlet per le previsioni meteo, per la geolocalizzazione, per l'inserimento di dati,.. \\
	Queste portlet, in quanto finestre, sono adattabili alle esigenze del singolo utente e possono quindi esse chiuse, allargate/ridotte, spostate}
}
\newglossaryentry{middleware}
{
	name=\glslink{middleware}{MIDDLEWARE},
	text=middleware,
	sort=middleware,
	description={con il termine \emph{middleware} si intende un insieme di programmi informatici che fungono da intermediari tra diverse apllicazioni e componenti software}
}

\newglossaryentry{urlg}
{name=\glslink{url}{URL},
text=URL,
sort=url,
description={un URL, Uniform Resource Locator, è l'indirizzo univoco in cui si trova una risorsa, come ad esempio un documento, immagine o video all'interno della rete internet}
}

\newglossaryentry{record}
{name=\glslink{record}{RECORD},
	text=record,
	sort=record,
	description={Nel contesto corrente, per record si intende l'insieme dei dati inseriti da un utente, nel caso di una form per l'inserimento di una nuova scheda, oppure dell' insieme dati, recuperati dal database, riguardanti una particolare scheda}
}

\newglossaryentry{htmlg}
{name=\glslink{html}{HTML},
	text=HTML,
	sort=html,
	description={HTML, HyperText Markup Language, è il principale linguaggio utilizzato per rappresentare le pagine web nella rete internet. è definito come linguaggio di markup, nel senso che definisce, tamiti elementi definiti tag, le modalità con cui un browser deve rappresentare la porzione di testo racchiusa all'interno del tag}
}

\newglossaryentry{ajaxg}
{name=\glslink{ajax}{AJAX},
	text=AJAX,
	sort=ajax,
	description={AJAX, acronimo di \emph{Asyncronous JavaScript And XML}, è una tecnica di sviluppo software per la realizzazione di applicazioni web interattive. Lo sviluppo di applicazioni HTML con AJAX si basa su uno scambio di dati in \emph{background} fra web browser e server, che consente l'aggiornamento dinamico di una pagina web senza esplicito ricaricamento da parte dell'utente}
}

\newglossaryentry{ictg}
{name=\glslink{ict}{ICT},
	text=ICT,
	sort=ict,
	description={Per ICT, acronimo di \emph{Information and Communication Technologies}, si intende l'insieme dei metodi e delle tecniche utilizzate nella trasmissione, ricezione ed elaborazione di informazioni (tecnologie web e digitali comprese), ampiamente diffusi a partire dagli anni '60 del secolo scorso ed oggi permeati praticamente in ogni aspetto della nostra vita}
}

\newglossaryentry{framework}
{name=\glslink{framework}{FRAMEWORK},
	text=framework,
	sort=framework,
	description={Un framework, letteralmente \emph{intelaiatura}, è un'architettura logica di supporto (spesso un'implementazione logica di un particolare design pattern) su cui un software può essere progettato e realizzato, spesso facilitandone lo sviluppo da parte del programmatore}
}

\newglossaryentry{servlet}
{name=\glslink{servlet}{SERVLET},
	text=servlet,
	sort=servlet,
	description={Una servlet èoggetto scritto in linguaggio Java che operano all'interno di un server web (es. Tomcat, Jetty) oppure un server per applicazioni (es. WildFly, GlassFish), permettendo la creazione di applicazione web}
}

\newglossaryentry{restg}
{name=\glslink{rest}{REST},
	text=REST,
	sort=rest,
	description={REST (REpresentational State Transfert) definisce un insieme di principi architetturali per la progettazione di un sistema. Questi princìpi comprendono:
	\begin{itemize}
		\item identificazione delle risorse;
		\item utilizzo esplicito dei metodi HTTP;
		\item risorse autodescrittive;
		\item collegamenti tra risorse;
		\item comunicazione senza stato.
	\end{itemize}}
}