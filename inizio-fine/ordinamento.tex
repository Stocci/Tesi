\cleardoublepage
\phantomsection
\pdfbookmark{Sommario}{Sommario}
\begingroup
\let\clearpage\relax
\let\cleardoublepage\relax
\let\cleardoublepage\relax


\chapter*{Organizzazione del testo}

\begin{description}
    \item[{\hyperref[cap:introduzione]{Il primo capitolo}}] descrive l'azienda ospitante ed il contesto in cui si è svolto lo stage.

    \item[{\hyperref[cap:tecnologie-strumenti]{Il secondo capitolo}}] enuncia e tecnologie e gli strumenti con cui ho dovuto relazionarmi durante lo sviluppo del progetto;
    
    \item[{\hyperref[cap:descrizione-stage]{Il terzo capitolo}}] approfondisce il dominio applicativo ed i casi d'uso del progetto.
    
    \item[{\hyperref[cap:progettazione]{Il quarto capitolo}}] approfondisce l'architettura del software JGalileo CRM in generale, ed in particolare la portlet relativa alla form.    
    \item[{\hyperref[cap:sviluppo-prototipo]{Il quinto capitolo}}] descrive la logica adottata per la realizzazione del prototipo oggetto dell'esperienza di stage ed i risultati ottenuti
    \item[{\hyperref[cap:conclusioni]{Nel capitolo conclusivo}}] vengono tratte le conclusioni, sia oggettive che personali, dell'esperienza di stage.
  
\end{description}

Riguardo la stesura del testo, relativamente al documento sono state adottate le seguenti convenzioni tipografiche:
\begin{itemize}
	\item gli acronimi, le abbreviazioni e i termini ambigui o di uso non comune menzionati vengono definiti nel glossario, situato alla fine del presente documento;
	\item per la prima occorrenza dei termini riportati nel glossario viene utilizzata la seguente nomenclatura: \emph{parola}\glsfirstoccur;
	\item i termini in lingua straniera o facenti parti del gergo tecnico sono evidenziati con il carattere \emph{corsivo}.
\end{itemize}